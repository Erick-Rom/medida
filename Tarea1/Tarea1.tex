\documentclass[12pt]{article}
\usepackage[spanish]{babel}
\usepackage{amssymb}
\usepackage{amsmath}
\usepackage{geometry}
 \geometry{
 letterpaper,
 total={170mm,237mm},
 left=20mm,
 top=15mm,
 }
\usepackage{setspace}
\spacing{1.5}
\setlength{\parindent}{0pt}
%Falta agregar una portada bien hecha aparte.
\LARGE{\title{Tareas de primer parcial-Topología}}
\author{Alumnos: \\
Erick Román Montemayor Treviño - 1957959 \\
Renata - 1956093\\
Diego - \\
Everardo Flores Rivera - 2127301}
\begin{document}
\maketitle

\paragraph{1}
\textit{Sea X un conjunto y $\mathcal{A} $ un anillo sobre X. Sea $\{A_n\}_{n\in\mathbb{N}}$. Demostrar por inducción que $\bigcap\limits_{n\in\mathbb{N}}A_n\in\mathcal{A} $}

El caso base fue demostrado en clase, ahora partiendo de la hipótesis inductiva, esto es:

$\bigcap\limits_{n=1}^{n=k}A_n\in\mathcal{A}$ veamos que se cumple para $n=k+1$

Podemos concluir que 
$\bigcap\limits_{n=1}^{n=k+1}A_n=\bigcap\limits_{n=1}^{n=k}A_n\bigcap\limits_{n=1}A_n\in\mathcal{A}$ por la hipótesis inicial, la hipótesis inductiva y el caso base.
\paragraph{2}
\textit{Hallar un anillo que no sea algebra.}

Sea $X=\mathbb{Z}$, $S=\{\varnothing\}\cup P(2\mathbb{Z})$, es fácil ver que S es anillo pero no es un álgebra ya que $X\notin S$

\paragraph{3}
\textit{Sean X, Y conjuntos y $S_y$ una $\sigma-$álgebra en Y. Sea $f:X\rightarrow Y$, demuestre que la colección $\{f^{-1}(B):B\in S_y\}$ es una $\sigma-$álgebra }
\\
Sea $\mathcal{A} = \{f^{-1}(B) : B \in S_y\}$. Para demostrar que $\mathcal{A}$ es una $\sigma$-álgebra en $X$, debemos verificar tres propiedades:

\begin{enumerate}
    \item \textbf{El conjunto vacío está en $\mathcal{A}$:}
    Como $S_y$ es una $\sigma$-álgebra en $Y$, por definición $\emptyset \in S_y$. Dado que la preimagen del conjunto vacío es el conjunto vacío, tenemos que $f^{-1}(\emptyset) = \emptyset$. Por lo tanto, $\emptyset \in \mathcal{A}$.

    \item \textbf{Cerradura bajo complementos:}
    Sea $A \in \mathcal{A}$. Por definición de nuestra colección, existe un conjunto $B \in S_y$ tal que $A = f^{-1}(B)$. El complemento de $A$ en $X$ es $A^c = X \setminus A = X \setminus f^{-1}(B)$.
    
    Por propiedades de la preimagen, sabemos que el complemento de una preimagen es la preimagen del complemento:
    $$X \setminus f^{-1}(B) = f^{-1}(Y \setminus B)$$
    
    Como $B \in S_y$ y $S_y$ es una $\sigma$-álgebra, su complemento $B^c = Y \setminus B \in S_y$. Al ser la preimagen de un elemento de $S_y$, concluimos que $A^c = f^{-1}(B^c) \in \mathcal{A}$.

    \item \textbf{Cerradura bajo uniones numerables:}
    Sea $\{A_n\}_{n \in \mathbb{N}} \subset \mathcal{A}$ una sucesión numerable de conjuntos. Para cada $n \in \mathbb{N}$, existe $B_n \in S_y$ tal que $A_n = f^{-1}(B_n)$.
    
    La unión de la sucesión es:
    $$\bigcup_{n \in \mathbb{N}} A_n = \bigcup_{n \in \mathbb{N}} f^{-1}(B_n)$$
    
    Por propiedades de la preimagen, la unión de preimágenes es la preimagen de la unión:
    $$\bigcup_{n \in \mathbb{N}} f^{-1}(B_n) = f^{-1}\left(\bigcup_{n \in \mathbb{N}} B_n\right)$$
    
    Dado que $S_y$ es una $\sigma$-álgebra y $\{B_n\}_{n \in \mathbb{N}} \subset S_y$, por definición sabemos que $\bigcup_{n \in \mathbb{N}} B_n \in S_y$. 
    Por lo tanto, $\bigcup_{n \in \mathbb{N}} A_n \in \mathcal{A}$.
\end{enumerate}

Al cumplir las tres propiedades, concluimos que $\mathcal{A}$ es una $\sigma$-álgebra en $X$.
\end{proof}


\paragraph{4}
\textit{Demostrar que $(0,1]=\bigcap\limits_{n\in\mathbb{N}}(0,1+\frac{1}{n})$.}

Veamos que $(0,1]\subset \bigcap\limits_{n\in\mathbb{N}}(0,1+\frac{1}{n})$.

Sea $x\in(0,1]$, esto es $0<x\leq1$ , se sigue que $\forall n\in\mathbb{N}, 0<x\leq1<1+\frac{1}{n}$

Entonces $x\in(0,1+\frac{1}{n}) \forall n\in\mathbb{N}$

Entonces $x\in\bigcap\limits_{n\in\mathbb{N}}(0,1+\frac{1}{n})$

Ahora veamos la otra contención. Esto es que $\bigcap\limits_{n\in\mathbb{N}}(0,1+\frac{1}{n})\subset (0,1]$


Sea $x\in\bigcap\limits_{n\in\mathbb{N}}(0,1+\frac{1}{n})$, esto es: $\forall n\in\mathbb{N}, 0<x<1+\frac{1}{n}$. 

De aqui es evidente que $0<x\leq1$. Ahora veamos que $x\ngtr 1$

Supongamos que $x>1\Rightarrow x=1+\frac{1}{n_0}$ para algun $n_0\in\mathbb{N}$. Ahora por principio Arquimideano tenemos lo siguiente: $\exists N>n_0 :\space 1+\frac{1}{N}<1+\frac{1}{n_0}=x$. Entonces $x\notin\bigcap\limits_{n\in\mathbb{N}}(0,1+\frac{1}{n})$ lo cual es una contracción.

Esto demuestra la igualdad de los conjuntos.
\paragraph{5}
\textit{Demostrar las 5 propiedades de $F_\sigma$ y $G_\delta$}
\begin{itemize}
    \item $F_\sigma$ y $G_\delta$ son bolerianos
    
    Sabemos que todos los conjuntos abiertos son Bolerianos por la definición, ademas sabemos que la intersección numerable de conjuntos de una $\sigma-algebra$ es elemento de la $\sigma-algebra$, así $G_\delta$ es Boleriano. Analogamente todos los cerrados son bolerianos y por definición de $\sigma-algebra$ la union numerablede cerrados también, asi $F_\sigma$ es Boleriano.

    \item Si A es $F_\sigma$, entonces $A^c=G_\delta$
    

    \item Si A es $G_\delta$, entonces $A^c=F_\sigma$
    \item La union numerable de conjuntos $F_\sigma$ es $F_\sigma$
    \item La intersección numerable de conjuntos $G_\delta$ es $G_\delta$
\end{itemize}

\paragraph{6}
\textit{Sea X conjunto y S $\sigma-algebra$. Sean $x_0\in X; E\in S$. Definimos $\mu_{x_0}:S\rightarrow [0,\infty]$ Como \(\mu_{x_0}(E)=
\begin{cases}
1, & x_0\in E \\
0,  & x_0\notin E
\end{cases}
\)\\ Demostrar que $\mu_{x_0}$ es medida.}

\paragraph{7}
\textit{Sea $(X,\mu, S)$ un esp. de medida y ${E_n}_{n\in\mathbb{N}}$ una suc decreciente con respecto la contención en S. Si $\mu (E_1)<\infty$, demuestre que $\mu (\bigcap\limits_{n\in\mathbb{N}}E_n)=\lim\limits_{n\to\infty}\mu(E_n)$}

Se sabe que si $\{F_n\}\subset S$ es sucesión creciente, entonces
\[\mu\left(\bigcup_{n=1}^{\infty} F_n\right)=\lim_{n\to\infty}\mu(F_n).\]

Sea la sucesión $G_n$ tal que
\[G_n=E_1-E_n.\]

P.D. $\{G_n\}_{n\in\mathbb{N}}$ es sucesión creciente.

Si tomamos $G_1$ y $G_2$,

\[G_1=E_1-E_1=\varnothing,\]
\[G_2=E_1-E_2.\]

Pero $E_2\subset E_1$, al menos que $E_1=E_2$, por ende
\[G_1\subset G_2\]

Supongamos que
\[G_1\subset G_2\subset \cdots \subset G_n.\]

P.D.
\[G_1\subset G_2\subset \cdots \subset G_n \subset G_{n+1}\]

\[G_{n+1}=E_1-E_{n+1}\]
\[G_n=E_1-E_n.\]

Tenemos que
\[E_{n+1}\subset E_n \subset E_1,\]

por conjuntos tenemos que
\[E_1-E_{n+1} \supset E_1-E_n\Rightarrow G_{n+1}\supset G_n.\]

Por lo tanto
\[G_1\subset G_2\subset \cdots \subset G_n\subset G_{n+1}.\]

es decir, $\{G_n\}_{n\in\mathbb{N}}$ es creciente respecto a contención.

Entonces
\[\mu\left(\bigcup_{n=1}^{\infty} G_n\right)=\lim_{n\to\infty}\mu(G_n).\]

\[\mu\left(\bigcup_{n=1}^{\infty} (E_1-E_n)\right)=\lim_{n\to\infty} \mu(E_1-E_n).\]

\[\mu\left(E_1-\bigcap_{n=1}^{\infty} E_n\right)=
\lim_{n\to\infty} \left[\mu(E_1)-\mu(E_n)\right]\quad \text{por teorema de conjuntos y un teorema de medida}\]

\[\mu(E_1)-\mu\left(\bigcap_{n=1}^{\infty} E_n\right)=\mu(E_1)-\lim_{n\to\infty} \mu(E_n).\]
por lo tanto 
\[\mu\left(\bigcap_{n=1}^{\infty} E_n\right)=\lim_{n\to\infty} \mu(E_n).\]

\paragraph{8}
\textit{Sea $t \in \mathbb{R}$ y $A \subset \mathbb{R}$. Se define el conjunto\[t\cdot A = \{\, t a : a \in A \,\}.\] y se conoce como dilatcación de A. Demostrar que $m^{*}(t\cdot A)=|t| \cdot m^{*}(A)$}
Sea la coleccion $\{{I_n}\}_{n\in \mathbb{N}},$ numerable tal que  $A\subset \bigcup_{k=1}^\infty I_n$ entonces $t\cdot A \subset\bigcup_{k=1}^\infty t\cdot I_n$.
Se tiene que 
\[
m^{*}(t\cdot A) \le \sum_{n \in \mathbb{N}}|  t\cdot I_n |=|t|\cdot \sum_{n \in \mathbb{N}}|I_n |
\]
entonces $m^{*}(t\cdot A) \le |t| \cdot m^{*}(A)$ esto es por ínfimo
para la otra desigualdad
\[
m^{*}(A)=m^{*}((1/t)\cdot tA) \le |1/t|m^{*}(t\cdot A)
\]
el último paso es por la demostración de la primera parte, y mulplicando por t tenemos que $|t|\cdot m^{*}(A)\le m^{*}(t\cdot A)$. 

Por lo tanto $|t|\cdot m^{*}(A)= m^{*}(t\cdot A)$

\paragraph{9}
\textit{Sean $a,b,c,d\in\mathbb{R}\space : a<b \space\wedge c<d$.\\
Demostrar que $m^*((a,b)\cup(c,d))=(b-a)+(d-c)$ ssi $(a,b)\cap(c,d)\neq\emptyset $}

$(\Rightarrow)$ Si $(a,b)\cap(c,d)\neq\varnothing$, por ende a,b,c,d son todos diferentes a lo mucho uno es igual a 0 entonces\[
(a,b)\cup(c,d)=(\min(a,c),\max(b,d)).
\] y
\[m^*((\min(a,c),\max(b,d)))=\max(b,d)-\min(a,c).\]

Se tiene que
\[\max(b,d)\le b+d\qquad \text{y} \qquad\min(a,c)\ge a+c.\]

Luego
\[
\max(b,d)-\min(a,c)<b+d-a-c=(b-a)+(d-c).\]

\[\Rightarrow m^*((a,b)\cup(c,d))<(b-a)+(d-c).\]

por lo tanto
\[m^*((a,b)\cup(c,d))=(b-a)+(d-c)\Rightarrow(a,b)\cap(c,d)=\varnothing.\]
$(\Leftarrow)$
\[m^*((a,b)\cup(c,d))\le (b-a)+(d-c)\]
por subaditividad.

Sean $I_1=(a,b)$ y $I_2=(c,d)$.

Como $I_1\cap I_2=\varnothing$ y $I_1\cup I_2\subset \bigcup J_n$,

\[\sum |J_n| \ge (b-a)+(d-c),\] porque

\[\sum |J_n|\ge (b-a)\qquad \text{y} \qquad\sum |J_n|\ge (d-c).\]

\[m^*(I_1\cup I_2)\ge (b-a)+(d-c)\]
por definición de ínfimo. Por lo tanto
\[m^*((a,b)\cup(c,d))=(b-a)+(d-c).\]\[\Longleftrightarrow(a,b)\cap(c,d)=\varnothing.\]

\paragraph{10}
\textit{Si $A\in\mu$. Demostrar que $t+E\in\mu$ $\forall t\in\mathbb{R}$}

\paragraph{11}
\textit{Demostrar que $E\subset\mathbb{R}$ es medible ssi $\forall\epsilon>0\exists U$ abto. en  $\mathbb{R}$ t.q $C\subset E\subset U \wedge m(U-C)<\epsilon$}

$\Rightarrow$
Si $E$ es medible se sabe que si $\varepsilon/2>0$ $\exists U$ abierto en $\mathbb{R}$ tq. $E \subset U \Rightarrow m(U-E)<\varepsilon/2$ y 
$C$ cerrado tq. $C \subset E \Rightarrow m(C-E)<\varepsilon/2$.

Sabemos que

\[m(U-C)=m((U-E)\cup(E-C))\le m(U-E)+m(E-C) \text{ por subaditividad}\]

\[<\varepsilon/2+\varepsilon/2=\varepsilon\]

\[\therefore m(U-C)<\varepsilon\]

$\Leftarrow$ P.D.\[m(E)=m(U\cap E)+m(U\cap E^c), \quad E\subset U\]

\[E=(U\cap E)\cup(U\cap E^c)\text{ por conjuntos}\]

\[m(E)\le m(U\cap E)+m(U\cap E^c)\text{ por subaditividad}\]

\[m(U\cap E)+m(U\cap E^c)=m(E)+m(U-E)<m(E)+\varepsilon\text{ por igualdad de conjuntos y la hipótesis}\]

\[\therefore m(U\cap E)+m(U\cap E^c)\le m(E) \text{ por ser epsilon arbritario}\]

\[\therefore m(E)=m(U\cap E)+m(U\cap E^c)\]

\[\therefore E \text{ es medible.}\]

\paragraph{12}
\textit{Sea $E\subset[0,1]$ medible con $m(E)>0$. Demostrar $\exists x,y\in E : x>y$ tales que $x-y\in\mathbb{Q}^c$}

Suponemos que $x-y \in \mathbb{Q} \ \forall x,y \in E$ \[\Rightarrow x-y=q \in \mathbb{Q} \Rightarrow x=y+q \ \forall x,y \in E,\ q\in\mathbb{Q}\]

\[\Rightarrow x \in y+\mathbb{Q} \ \text{o decir} \ E \subset y+\mathbb{Q},\ y\in E\]

\[\Rightarrow m(E) \le m(y+\mathbb{Q}) = 0 \quad \text{medida invariante bajo traslaciones}\]

\[\Rightarrow m(E)=0 \text{ contradicción}\]

\[\therefore \exists x,y \in E \ \text{t.q.} \ x-y \in \mathbb{Q}^c\]

\paragraph{13}
\textit{Sea V el conjunto de Vitali si $E\in\mu$, demostrar que $m(E)=0$.}

\paragraph{14}
\textit{Demostrar que $\mathbb{R}-\bigcup\limits_{n\in\mathbb{N}}(r_n-\frac{1}{n^2},r_n+\frac{1}{n^2})\neq\emptyset$ donde $r_n\in\mathbb{Q}$}
\end{document}
