\documentclass[12pt]{article}
\usepackage[spanish]{babel}
\usepackage{amssymb}
\usepackage{amsmath}
\usepackage{geometry}
 \geometry{
 letterpaper,
 total={170mm,237mm},
 left=20mm,
 top=15mm,
 }
\usepackage{setspace}
\spacing{1.5}
\setlength{\parindent}{0pt}
%Falta agregar una portada bien hecha aparte.
\LARGE{\title{Tareas de primer parcial-Topología}}
\author{Alumnos: \\
Erick Román Montemayor Treviño - 1957959 \\
Renata - 1956093\\
Diego - \\
Everardo Flores Rivera - 2127301}
\begin{document}
\maketitle

\paragraph{1}
\textit{Sea X un conjunto y $\mathcal{A} $ un anillo sobre X. Sea $\{A_n\}_{n\in\mathbb{N}}$. Demostrar por inducción que $\bigcap\limits_{n\in\mathbb{N}}A_n\in\mathcal{A} $}

El caso base fue demostrado en clase, ahora partiendo de la hipótesis inductiva, esto es:

$\bigcap\limits_{n=1}^{n=k}A_n\in\mathcal{A}$ veamos que se cumple para $n=k+1$

Podemos concluir que 
$\bigcap\limits_{n=1}^{n=k+1}A_n\in\mathcal{A}=\bigcap\limits_{n=1}^{n=k}A_n\bigcap\limits_{n=1}A_n\in\mathcal{A}$ por la hipótesis inicial, la hipótesis inductiva y el caso base.
\paragraph{2}
\textit{Hallar un anillo que no sea algebra.}

Sea $X=\mathbb{Z}$, $S=\{\varnothing\}\cup P(2\mathbb{Z})$, es fácil ver que S es anillo pero no es un álgebra ya que $X\notin S$

\paragraph{3}
\textit{Sean X, Y conjuntos y $S_y$ una $\sigma-$álgebra en Y. Sea $f:X\rightarrow Y$, demuestre que la colección $\{f^{-1}(B):B\in S_y\}$ es una $\sigma-$álgebra }

\paragraph{4}
\textit{Demostrar que $(0,1]=\bigcap\limits_{n\in\mathbb{N}}(0,1+\frac{1}{n})$.}

Veamos que $(0,1]\subset \bigcap\limits_{n\in\mathbb{N}}(0,1+\frac{1}{n})$.

Sea $x\in(0,1]$, esto es $0<x\leq1$ , por propiedad arquimediana existe $ N \in\mathbb{N}\hspace{4px}t.q \hspace{4px} \forall n\geq N \hspace{4px} 0< \frac{1}{n} \leqslant  x<1$ entonces $x\in[\frac{1}{n},1)$
entonces $x\in\bigcup\limits_{n\in\mathbb{N}-\{1\}}[\frac{1}{n},1)$\\Por tanto $(0,1)\subset\bigcup\limits_{n\in\mathbb{N}-\{1\}}[\frac{1}{n},1)$

Ahora veamos la otra contención.

Sea $x\in\bigcup\limits_{n\in\mathbb{N}-\{1\}}[\frac{1}{n},1)$, entonces $0<\frac{1}{n_0}\leq x<1$ para algun $n_0\in\mathbb{N}-\{1\}$\\$x\in(0,1)$\\ Por tanto $(0,1)\supset\bigcup\limits_{n\in\mathbb{N}-\{1\}}[\frac{1}{n},1)$

Esto demuestra la igualdad de los conjuntos.
\paragraph{5}
\textit{Demostrar las 5 propiedades de $F_\sigma$ y $G_\delta$}
\begin{itemize}
    \item $F_\sigma$ y $G_\delta$ son bolerianos
    
    Sabemos que todos los conjuntos abiertos son Bolerianos por la definición, ademas sabemos que la intersección numerable de conjuntos de una $\sigma-algebra$ es elemento de la $\sigma-algebra$, así $G_\delta$ es Boleriano. Analogamente todos los cerrados son bolerianos y por definición de $\sigma-algebra$ la union numerablede cerrados también, asi $F_\sigma$ es Boleriano.

    \item Si A es $F_\sigma$, entonces $A^c=G_\delta$
    

    \item Si A es $G_\delta$, entonces $A^c=F_\sigma$
    \item La union numerable de conjuntos $F_\sigma$ es $F_\sigma$
    \item La intersección numerable de conjuntos $G_\delta$ es $G_\delta$
\end{itemize}

\paragraph{6}
\textit{Sea X conjunto y S $\sigma-algebra$. Sean $x_0\in X; E\in S$. Definimos $\mu_{x_0}:S\rightarrow [0,\infty]$ Como \(\mu_{x_0}(E)=
\begin{cases}
1, & x_0\in E \\
0,  & x_0\notin E
\end{cases}
\)\\ Demostrar que $\mu_{x_0}$ es medida.}



\end{document}