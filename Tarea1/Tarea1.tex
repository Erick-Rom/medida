\documentclass[12pt]{article}
\usepackage[spanish]{babel}
\usepackage{amssymb}
\usepackage{amsmath}
\usepackage{geometry}
 \geometry{
 letterpaper,
 total={170mm,237mm},
 left=20mm,
 top=15mm,
 }
\usepackage{setspace}
\spacing{1.5}
\setlength{\parindent}{0pt}
%Falta agregar una portada bien hecha aparte.
\LARGE{\title{Tareas de primer parcial-Topología}}
\author{Alumnos: \\
Erick Román Montemayor Treviño - 1957959 \\
Renata - 1956093\\
Diego - \\
Everardo Flores Rivera - 2127301}
\begin{document}
\maketitle

\paragraph{1}
\textit{Sea X un conjunto y $\mathcal{A} $ un anillo sobre X. Sea $\{A_n\}_{n\in\mathbb{N}}$. Demostrar por inducción que $\bigcap\limits_{n\in\mathbb{N}}A_n\in\mathcal{A} $}

El caso base fue demostrado en clase, ahora partiendo de la hipótesis inductiva, esto es:

$\bigcap\limits_{n=1}^{n=k}A_n\in\mathcal{A}$ veamos que se cumple para $n=k+1$

Podemos concluir que 
$\bigcap\limits_{n=1}^{n=k+1}A_n=\bigcap\limits_{n=1}^{n=k}A_n\bigcap\limits_{n=1}A_n\in\mathcal{A}$ por la hipótesis inicial, la hipótesis inductiva y el caso base.
\paragraph{2}
\textit{Hallar un anillo que no sea algebra.}

Sea $X=\mathbb{Z}$, $S=\{\varnothing\}\cup P(2\mathbb{Z})$, es fácil ver que S es anillo pero no es un álgebra ya que $X\notin S$

\paragraph{3}
\textit{Sean X, Y conjuntos y $S_y$ una $\sigma-$álgebra en Y. Sea $f:X\rightarrow Y$, demuestre que la colección $\{f^{-1}(B):B\in S_y\}$ es una $\sigma-$álgebra }



\paragraph{4}
\textit{Demostrar que $(0,1]=\bigcap\limits_{n\in\mathbb{N}}(0,1+\frac{1}{n})$.}

Veamos que $(0,1]\subset \bigcap\limits_{n\in\mathbb{N}}(0,1+\frac{1}{n})$.

Sea $x\in(0,1]$, esto es $0<x\leq1$ , se sigue que $\forall n\in\mathbb{N}, 0<x\leq1<1+\frac{1}{n}$

Entonces $x\in(0,1+\frac{1}{n}) \forall n\in\mathbb{N}$

Entonces $x\in\bigcap\limits_{n\in\mathbb{N}}(0,1+\frac{1}{n})$

Ahora veamos la otra contención. Esto es que $\bigcap\limits_{n\in\mathbb{N}}(0,1+\frac{1}{n})\subset (0,1]$


Sea $x\in\bigcap\limits_{n\in\mathbb{N}}(0,1+\frac{1}{n})$, esto es: $\forall n\in\mathbb{N}, 0<x<1+\frac{1}{n}$. 

De aqui es evidente que $0<x\leq1$. Ahora veamos que $x\ngtr 1$

Supongamos que $x>1\Rightarrow x=1+\frac{1}{n_0}$ para algun $n_0\in\mathbb{N}$. Ahora por principio Arquimideano tenemos lo siguiente: $\exists N>n_0 :\space 1+\frac{1}{N}<1+\frac{1}{n_0}=x$. Entonces $x\notin\bigcap\limits_{n\in\mathbb{N}}(0,1+\frac{1}{n})$ lo cual es una contracción.

Esto demuestra la igualdad de los conjuntos.
\paragraph{5}
\textit{Demostrar las 5 propiedades de $F_\sigma$ y $G_\delta$}
\begin{itemize}
    \item $F_\sigma$ y $G_\delta$ son bolerianos
    
    Sabemos que todos los conjuntos abiertos son Bolerianos por la definición, ademas sabemos que la intersección numerable de conjuntos de una $\sigma-algebra$ es elemento de la $\sigma-algebra$, así $G_\delta$ es Boleriano. Analogamente todos los cerrados son bolerianos y por definición de $\sigma-algebra$ la union numerablede cerrados también, asi $F_\sigma$ es Boleriano.

    \item Si A es $F_\sigma$, entonces $A^c=G_\delta$
    

    \item Si A es $G_\delta$, entonces $A^c=F_\sigma$
    \item La union numerable de conjuntos $F_\sigma$ es $F_\sigma$
    \item La intersección numerable de conjuntos $G_\delta$ es $G_\delta$
\end{itemize}

\paragraph{6}
\textit{Sea X conjunto y S $\sigma-algebra$. Sean $x_0\in X; E\in S$. Definimos $\mu_{x_0}:S\rightarrow [0,\infty]$ Como \(\mu_{x_0}(E)=
\begin{cases}
1, & x_0\in E \\
0,  & x_0\notin E
\end{cases}
\)\\ Demostrar que $\mu_{x_0}$ es medida.}

\paragraph{7}
\textit{Sea $(X,\mu, S)$ un esp. de medida y ${E_n}_{n\in\mathbb{N}}$ una suc decreciente con respecto la contención en S. Si $\mu (E_1)<\infty$, demuestre que $\mu (\bigcap\limits_{n\in\mathbb{N}}E_n)=\lim\limits_{n\to\infty}\mu(E_n)$}

\paragraph{8}
\textit{Sea $t \in \mathbb{R}$ y $A \subset \mathbb{R}$. Se define el conjunto\[t\cdot A = \{\, t a : a \in A \,\}.\] y se conoce como dilatcación de A. Demostrar que $m^{*}(t\cdot A)=|t| \cdot m^{*}(A)$}
Sea la coleccion $\{{I_n}\}_{n\in \mathbb{N}},$ numerable tal que  $A\subset \bigcup_{k=1}^\infty I_n$ entonces $t\cdot A \subset\bigcup_{k=1}^\infty t\cdot I_n$.
Se tiene que 
\[
m^{*}(t\cdot A) \le \sum_{n \in \mathbb{N}}|  t\cdot I_n |=|t|\cdot \sum_{n \in \mathbb{N}}|I_n |
\]
entonces $m^{*}(t\cdot A) \le |t| \cdot m^{*}(A)$ esto es por ínfimo
para la otra desigualdad
\[
m^{*}(A)=m^{*}((1/t)\cdot tA) \le |1/t|m^{*}(t\cdot A)
\]
el último paso es por la demostración de la primera parte, y mulplicando por t tenemos que $|t|\cdot m^{*}(A)\le m^{*}(t\cdot A)$. 

Por lo tanto $|t|\cdot m^{*}(A)= m^{*}(t\cdot A)$

\paragraph{9}
\textit{Sean $a,b,c,d\in\mathbb{R}\space : a<b \space\wedge c<d$.\\
Demostrar que $m^*((a,b)\cup(c,d))=(b-a)+(d-c)$ ssi $(a,b)\cap(c,d)\neq\emptyset $}

\paragraph{10}
\textit{Si $A\in\mu$. Demostrar que $t+E\in\mu$ $\forall t\in\mathbb{R}$}

\paragraph{11}
\textit{Demostrar que $E\subset\mathbb{R}$ es medible ssi $\forall\epsilon>0\exists U$ abto. en  $\mathbb{R}$ t.q $C\subset E\subset U \wedge m(U-C)<\epsilon$}

\paragraph{12}
\textit{Sea $E\subset[0,1]$ medible con $m(E)>0$. Demostrar $\exists x,y\in E : x>y$ tales que $x-y\in\mathbb{Q}^c$}

\paragraph{13}
\textit{Sea V el conjunto de Vitali si $E\in\mu$, demostrar que $m(E)=0$.}

\paragraph{14}
\textit{Demostrar que $\mathbb{R}-\bigcup\limits_{n\in\mathbb{N}}(r_n-\frac{1}{n^2},r_n+\frac{1}{n^2})\neq\emptyset$ donde $r_n\in\mathbb{Q}$}
\end{document}
